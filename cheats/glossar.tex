\section{R-Glossar}

\begin{longtable}{llp{0.5 \textwidth}}
\textbf{Aufruf} & \textbf{Name} & \textbf{Beschreibung} \\
\verb|help(func)| & Hilfe & Hilfe zu Funktionen \\
\verb|sum(x)| & Summe & Summe der Daten \\
\verb|mean(x)| & Mittelwert & arithmetischer Mittelwert der Daten \\
\verb|sd(x)| & Standardabweichung & Standardabweichung der Daten ($\sigma$) \\
\verb|var(x)| & Varianz & Varianz der Daten 
  (\verb|var(x)| = \verb|sd(x)^2| = \verb|cov(x,x)|) \\
\verb|median(x)| & Median & 50 \% Quantil \\
\verb|quantile(x,prob,type=2)| & y \% Quantil & Quantil der Daten 
  \textbf{Achtung! type=2!} \\
\verb|length(x)| & Länge & Länge des Datenvektors \\
\verb|rep(x,times)| & Replicate & repliziert Elemente von Vektoren. Jedes 
  \verb|x| wird \verb|times| mal aufgelistet \\
\verb|cumsum(x)| & Kumulative Summe & Liefert einen Vektor, dessen Elemente die 
  Summe aller vorhergehenden Summe des Eingangsvektors sind. \\
\verb|scale(x)| & Standardisierung & Standardisiert Daten (mean=0, sd=1) \\
\verb|unique(x)| & "'Einzigartig"' & entfernt doppelt vorhandene Werte \\
\verb|apply(x)| & Anwenden & wendet eine Funktion auf jedes Element eines 
Vektors an. \\

\end{longtable}