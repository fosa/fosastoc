% coding:utf-8

%FOSASTOC, a LaTeX-Code for a electrical summary of stochastic
%Copyright (C) 2013, Daniel Winz, Ervin Mazlagic

%This program is free software; you can redistribute it and/or
%modify it under the terms of the GNU General Public License
%as published by the Free Software Foundation; either version 2
%of the License, or (at your option) any later version.

%This program is distributed in the hope that it will be useful,
%but WITHOUT ANY WARRANTY; without even the implied warranty of
%MERCHANTABILITY or FITNESS FOR A PARTICULAR PURPOSE.  See the
%GNU General Public License for more details.
%----------------------------------------

\chapter{R Grundlagen}
\newpage

\section{Vektoren \& Matrizen definieren}
\paragraph{Vektor mit Intervall 1 definieren}
\begin{Schunk}
\begin{Sinput}
> x <- 1:10; y <- 5:13
> x; y
\end{Sinput}
\begin{Soutput}
 [1]  1  2  3  4  5  6  7  8  9 10
\end{Soutput}
\begin{Soutput}
[1]  5  6  7  8  9 10 11 12 13
\end{Soutput}
\end{Schunk}

\paragraph{Vektor mit beliebigem festen Intevall definieren}
\begin{Schunk}
\begin{Sinput}
> x <- seq(from=1, to=10, by=2); y <- seq(1, 10, 2)
> x; y
\end{Sinput}
\begin{Soutput}
[1] 1 3 5 7 9
\end{Soutput}
\begin{Soutput}
[1] 1 3 5 7 9
\end{Soutput}
\end{Schunk}

\paragraph{Vektor manuell definieren}
\begin{Schunk}
\begin{Sinput}
> x <- c(77, 29, 12, 33, 4, 9, 809, -27)
> x
\end{Sinput}
\begin{Soutput}
[1]  77  29  12  33   4   9 809 -27
\end{Soutput}
\end{Schunk}

\paragraph{Matrix Spaltenweise definieren}
\begin{Schunk}
\begin{Sinput}
> a <- seq(1, 9, 1); m <- matrix(a, 3)
> m
\end{Sinput}
\begin{Soutput}
     [,1] [,2] [,3]
[1,]    1    4    7
[2,]    2    5    8
[3,]    3    6    9
\end{Soutput}
\end{Schunk}

\paragraph{Matrix Reihenweise definieren}
\begin{Schunk}
\begin{Sinput}
> a <- seq(1, 9, 1)
> m <- matrix(a, 3, byrow=TRUE)
> m
\end{Sinput}
\begin{Soutput}
     [,1] [,2] [,3]
[1,]    1    2    3
[2,]    4    5    6
[3,]    7    8    9
\end{Soutput}
\end{Schunk}

\section{Arithmetik}
\paragraph{Einfache Summen und Produkte}
\begin{Schunk}
\begin{Sinput}
> a=15; b=3
> a+b; a*b; a-b; a/b
\end{Sinput}
\begin{Soutput}
[1] 18
\end{Soutput}
\begin{Soutput}
[1] 45
\end{Soutput}
\begin{Soutput}
[1] 12
\end{Soutput}
\begin{Soutput}
[1] 5
\end{Soutput}
\end{Schunk}

\paragraph{Operationen auf Vektoren}
\begin{Schunk}
\begin{Sinput}
> a <- 1:10; b <- 2
> c <- b*a
> a
\end{Sinput}
\begin{Soutput}
 [1]  1  2  3  4  5  6  7  8  9 10
\end{Soutput}
\begin{Sinput}
> c
\end{Sinput}
\begin{Soutput}
 [1]  2  4  6  8 10 12 14 16 18 20
\end{Soutput}
\end{Schunk}

\paragraph{Operationen auf Matrizen}
\begin{Schunk}
\begin{Sinput}
> a <- seq(1, 9, 1); b <- 2
> m <- matrix(a, 3); n <- m*b
> m
\end{Sinput}
\begin{Soutput}
     [,1] [,2] [,3]
[1,]    1    4    7
[2,]    2    5    8
[3,]    3    6    9
\end{Soutput}
\begin{Sinput}
> n
\end{Sinput}
\begin{Soutput}
     [,1] [,2] [,3]
[1,]    2    8   14
[2,]    4   10   16
[3,]    6   12   18
\end{Soutput}
\end{Schunk}
