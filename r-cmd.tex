% coding:utf-8

%FOSASTOC, a LaTeX-Code for a electrical summary of stochastic
%Copyright (C) 2013, Daniel Winz, Ervin Mazlagic

%This program is free software; you can redistribute it and/or
%modify it under the terms of the GNU General Public License
%as published by the Free Software Foundation; either version 2
%of the License, or (at your option) any later version.

%This program is distributed in the hope that it will be useful,
%but WITHOUT ANY WARRANTY; without even the implied warranty of
%MERCHANTABILITY or FITNESS FOR A PARTICULAR PURPOSE.  See the
%GNU General Public License for more details.
%----------------------------------------

\chapter{R Grundlagen}
\lstinline{R} ist eine multiparadigmatische Programmiersprache für 
Statistik und wird als Teil des GNU-Projektes entwickelt. Die 
Besonderheit von R liegt in der Implementation vieler Algorithmen
und Analysen der Statistik aber auch von vielseitigen Möglichkeiten 
des Plottens. Diese Stärken und die Tatsache, dass \lstinline{R} 
freie Software ist, haben es zu einem beliebten Tool in Wissenschaft
und Industrie gemacht.

\newpage

\setkeys{Gin}{width=1\textwidth}

\section*{Informationen finden}
Dieses Kapitel soll eine kurze Zusammenfassung der wichtigsten 
Funktionen in \lstinline{R} liefern. Möchte man spezifische und
detaillierte Informationen so kann man diese mittels \lstinline{help()}
erhalten. 

\section{Hilfe}
\lstinline{R} bietet für jede Funktion eine Art Manpage an. 
Diese kann in der Konsole mittels \lstinline{help(name-der-Funktion)}
aufgerufen werden. 

\begin{Schunk}
\begin{Sinput}
> help(plot)
\end{Sinput}
\end{Schunk}

\begin{lstlisting}

\end{lstlisting}

\section{Vektoren \& Matrizen definieren}
\paragraph{Vektor mit Intervall 1 definieren}
\begin{Schunk}
\begin{Sinput}
> x <- 1:10; y <- 5:13
> x; y
\end{Sinput}
\begin{Soutput}
 [1]  1  2  3  4  5  6  7  8  9 10
\end{Soutput}
\begin{Soutput}
[1]  5  6  7  8  9 10 11 12 13
\end{Soutput}
\end{Schunk}

\paragraph{Vektor mit beliebigem festen Intevall definieren}
\begin{Schunk}
\begin{Sinput}
> x <- seq(from=1, to=10, by=2); y <- seq(1, 10, 2)
> x; y
\end{Sinput}
\begin{Soutput}
[1] 1 3 5 7 9
\end{Soutput}
\begin{Soutput}
[1] 1 3 5 7 9
\end{Soutput}
\end{Schunk}

\paragraph{Vektor manuell definieren}
\begin{Schunk}
\begin{Sinput}
> x <- c(77, 29, 12, 33, 4, 9, 809, -27)
> x
\end{Sinput}
\begin{Soutput}
[1]  77  29  12  33   4   9 809 -27
\end{Soutput}
\end{Schunk}

\paragraph{Matrix Spaltenweise definieren}
\begin{Schunk}
\begin{Sinput}
> a <- seq(1, 9, 1); m <- matrix(a, 3)
> m
\end{Sinput}
\begin{Soutput}
     [,1] [,2] [,3]
[1,]    1    4    7
[2,]    2    5    8
[3,]    3    6    9
\end{Soutput}
\end{Schunk}

\paragraph{Matrix Reihenweise definieren}
\begin{Schunk}
\begin{Sinput}
> a <- seq(1, 9, 1)
> m <- matrix(a, 3, byrow=TRUE)
> m
\end{Sinput}
\begin{Soutput}
     [,1] [,2] [,3]
[1,]    1    2    3
[2,]    4    5    6
[3,]    7    8    9
\end{Soutput}
\end{Schunk}

\section{Arithmetik}
\paragraph{Einfache Summen und Produkte}
\begin{Schunk}
\begin{Sinput}
> a=15; b=3
> a+b; a*b; a-b; a/b
\end{Sinput}
\begin{Soutput}
[1] 18
\end{Soutput}
\begin{Soutput}
[1] 45
\end{Soutput}
\begin{Soutput}
[1] 12
\end{Soutput}
\begin{Soutput}
[1] 5
\end{Soutput}
\end{Schunk}

\paragraph{Operationen auf Vektoren}
\begin{Schunk}
\begin{Sinput}
> a <- 1:10; b <- 2
> c <- b*a
> a
\end{Sinput}
\begin{Soutput}
 [1]  1  2  3  4  5  6  7  8  9 10
\end{Soutput}
\begin{Sinput}
> c
\end{Sinput}
\begin{Soutput}
 [1]  2  4  6  8 10 12 14 16 18 20
\end{Soutput}
\end{Schunk}

\paragraph{Operationen auf Matrizen}
\begin{Schunk}
\begin{Sinput}
> a <- seq(1, 9, 1); b <- 2
> m <- matrix(a, 3); n <- m*b
> m
\end{Sinput}
\begin{Soutput}
     [,1] [,2] [,3]
[1,]    1    4    7
[2,]    2    5    8
[3,]    3    6    9
\end{Soutput}
\begin{Sinput}
> n
\end{Sinput}
\begin{Soutput}
     [,1] [,2] [,3]
[1,]    2    8   14
[2,]    4   10   16
[3,]    6   12   18
\end{Soutput}
\end{Schunk}

\section{Plots}

\subsection{Gewöhnlicher Plot}
Um einen gewöhnlichen Plot zu erstellen kann \lstinline{plot()}
verwendet werden. 

\begin{Schunk}
\begin{Sinput}
> x <- c(1:20)
> y <- (runif(n=20))
> plot(x, y)
\end{Sinput}
\end{Schunk}

\subsubsection{Linien plotten}
Möchte man einen Plot ergänzen mit Linien kann nach \lstinline{plot()}
noch \lstinline{abline()} benutzt werden.

\begin{Schunk}
\begin{Sinput}
> x <- c(1:20)
> y <- (runif(n=20))
> plot(x, y)
> abline(h=mean(y))
\end{Sinput}
\end{Schunk}

\subsubsection{Segmente plotten}
Möchte man beispielsweise die Abweichung von Daten und Mittelwert
zeigen, kann \lstinline{segments()} benutzt werden. Dieses ist in der
Lage mehrere Liniensegmente zu einem Plot hinzuzufügen.

\begin{Schunk}
\begin{Sinput}
> x <- c(1:20)
> y <- (runif(n=20))
> plot(x, y)
> abline(h=mean(y))
> segments(x0=x, y0=mean(y), x1=x, y1=y)
\end{Sinput}
\end{Schunk}

\subsubsection{Flächen plotten}
Möchte man Rechtecke oder Flächen in einen Plot einfügen so kann man 
\lstinline{rect()} benutzen. Im folgenden ein Beispiel zur Darstellung
der Varianz.

\begin{Schunk}
\begin{Sinput}
> x <- c(1:20)
> y <- (runif(n=20, min=3, max=7))
> plot(x, y, ylim=c(0, 10))
> abline(h=mean(y))
> diff <- sqrt((y-mean(y))^2)
> rect(xleft=x, xright=(x+diff), ybottom=mean(y), ytop=y, col='gray')
\end{Sinput}
\end{Schunk}

\begin{figure}[h!]
\centering
\begin{subfigure}[b]{0.48\textwidth}
\includegraphics{r-cmd-014}
\caption{Gewöhnlicher Plot mit \lstinline{plot()}}
\end{subfigure}
\begin{subfigure}[b]{0.48\textwidth}
\includegraphics{r-cmd-015}
\caption{Linie mit \lstinline{abline()}}
\end{subfigure}

\begin{subfigure}[b]{0.48\textwidth}
\includegraphics{r-cmd-016}
\caption{Segmente mit \lstinline{segments()}}
\end{subfigure}
\begin{subfigure}[b]{0.48\textwidth}
\includegraphics{r-cmd-017}
\caption{Flächen mit \lstinline{rect()}}
\end{subfigure}

\end{figure}

\subsection{Boxplot}
