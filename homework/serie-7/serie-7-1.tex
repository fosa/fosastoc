\section{Aufgabe 1}

In der Stadt Zürich hat es bekanntlich viele Baustellen. Die Dauer $X$ der
Arbeiten bei einer Baustelle liege zwischen 0 und 20 Wochen. 
Die Dichte $f(x)$ habe die folgende Form (siehe Original).

\begin{enumerate}[(a)]
	\item Begründen Sie, warum $c=0.1$ ist und schreibe die Dichte 
          $f(x)$ explizit auf.
    \item Berechnen Sie die Wahrscheinlichkeit, dass die Bauzeit $X$
          weniger Betrage als 
          \begin{enumerate}[(i)]
              \item 5
              \item 10
          \end {enumerate}
    \item Skizzieren Die die kumulative Verteilungsfunktion.
    \item Berechnen Sie den Erwartungswert, den Median, und die
          Standardabweichung der Dauer $X$.
    \item $K=400 \cdot sqrt{X} $ entspreche dem Betrag in Franken, den die
          Arbeiten bei einer Baustelle kosten. Wie gross ist die 
          Wahrscheinlichkeit, dass die Arbeiten bei einer Baustelle
          höchstens 120'000.- Fr. kosten?
\end{enumerate}

\subsection*{a)}
\[ \int f(x) dx \stackrel{!}{=} 1 \] 
\[ \Rightarrow \text{bei 20 Wochen gilt: } 
   \frac{c\cdot20W}{2} = 1 \Rightarrow 
   c=\frac{1\cdot2}{20W} = \frac{2}{20} = \frac{1}{10} \]

\subsection*{b)}
\[ \int\limits_0^5 f(x) dx \stackrel{?}{=} \]
\[ y = m\cdot x + b \rightarrow y=f(x)=\frac{-0.1}{20}\cdot x + 0.1 \]
\subsubsection*{i}
\[ \int\limits_0^5\left( \frac{-5}{1000} \cdot x + 0.1 \right)dx =
   \left. 
   	\frac{-5}{1000} \cdot \frac{1}{2} \cdot x^2 + 0.1 \cdot x 
   \right|_0^5 = 0.4375 \]
\subsubsection*{ii}
\[ \int\limits_0^{10}\left( \frac{-5}{1000} \cdot x + 0.1 \right)dx =
   \left. 
   	\frac{-5}{1000} \cdot \frac{1}{2} \cdot x^2 + 0.1 \cdot x 
   \right|_0^{10} = 0.75 \]

\subsection*{c}
\emph{e-Funktion}

\subsection*{d}
\subsubsection*{Erwartungswert}
\[ E(X) = \int\limits_0^{20} x \cdot f(x) dx \quad \rightarrow \quad
   f(x) = \frac{-5}{1000} \cdot x + 0.1 \]
\[ \int\limits_0^{20} f(x) dx \stackrel{!}{=} \frac{1}{2} \cdot 1  \]
\[ \sum\limits_{i=0}^{x} f(x) = 
   \sum\limits_{i=0}^{x} \left( \frac{-5}{1000} \cdot i + 0.1 \right) \]
\[ Var(x) = E(x^2) - E(x)^2 = 
\int\limits_0^{20} x^2 
\left[ \frac{1}{10}\left(1-\frac{x}{20} \right) \right]dx 
- \left( \frac{20}{3} \right)^2
= \frac{200}{9}\]
\subsubsection*{Median}
\[ F(m) = 0.5 \Leftrightarrow \frac{m}{10} - 
   \frac{m^2}{10^2} \stackrel{!}{=} 0.5 \]
$m$ ist dabei das $q(0.5)$ bzw. das 50\% Quantil.
\[ \Rightarrow m = 5.858 \]
\subsubsection*{Standardabweichung}
\[ \sigma = \sqrt{Var(X)} = 4.71 \]

\subsection*{e}
\[ P\left[K \leq 120'000\right] =
   P\left[40'000\sqrt{x} \leq 120'000\right] \]
\[ \Rightarrow P\left[\sqrt{x}\leq\frac{120'000}{40'000}\right] =
   P\left[\sqrt{x}\leq3\right] =
   P\left[X \leq 9\right] =
   F(9) = 0.6975\]

\subsection*{f)}

\subsection*{g)}
